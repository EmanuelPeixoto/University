% Prof. Dr. Ausberto S. Castro Vera
% UENF - CCT - LCMAT - Curso de Ci\^{e}ncia da Computa\c{c}\~{a}o
% Campos, RJ,  2024 
% Disciplina: An\'{a}lise e Projeto de Sistemas
% Aluno: 
 

\chapterimage{ScalaH.jpg} % Table of contents heading image
\chapter{Considerações Finais}

Este trabalho apresentou o desenvolvimento de um sistema integrado para um estúdio de filmagem e edição profissional, focado em produções que utilizam tecnologias de ponta como câmera virtual e motion capture. O sistema foi projetado para atender diversos tipos de produções, desde comerciais até cenas para jogos digitais.

\section{Problemas Enfrentados}
Durante o desenvolvimento do projeto, diversos desafios técnicos precisaram ser considerados:

\begin{itemize}
    \item A necessidade de uma infraestrutura robusta de rede para suportar o tráfego intenso de dados entre os diferentes centros e o servidor central
    \item O gerenciamento eficiente do armazenamento de grandes volumes de dados provenientes das filmagens em alta resolução e scans 3D
    \item A complexidade da integração entre os diferentes sistemas (edição, filmagem, motion capture e scanning)
    \item A necessidade de backup e redundância para garantir a segurança dos dados dos projetos
    \item O desafio de criar uma interface web intuitiva para feedback dos clientes
\end{itemize}

\section{Síntese do Trabalho}
O sistema desenvolvido é composto por seis subsistemas principais:

\begin{enumerate}
    \item Central de Edição com 12 estações de trabalho de alta performance
    \item Centro de Filmagem equipado com tela LED infinita e sistema de câmera virtual
    \item Centro de Motion Capture com 8 trajes Rokoko e 25 câmeras base
    \item Centro de Scanning 3D com 100 câmeras LIDAR
    \item Data Center para armazenamento de backups e informações gerais
    \item Servidor Interno central para gerenciamento de projetos ativos
\end{enumerate}

Todos estes subsistemas foram integrados através de uma rede de alta velocidade e um sistema de gerenciamento de dados centralizado, permitindo o fluxo eficiente de trabalho entre as diferentes etapas de produção.

\section{Trabalhos Futuros}
Diversos aspectos poderiam ser explorados em trabalhos futuros para aprimorar ainda mais o sistema:

\begin{itemize}
    \item Implementação de um sistema de inteligência artificial para otimização do workflow e distribuição automática de recursos
    \item Desenvolvimento de um sistema de preview em tempo real das edições para clientes remotos
    \item Integração com sistemas de renderização em nuvem para aumentar a capacidade de processamento
    \item Implementação de um sistema de versionamento mais robusto para os projetos
    \item Desenvolvimento de aplicativos móveis para acompanhamento dos projetos
    \item Sistema de análise preditiva para manutenção dos equipamentos
    \item Expansão do sistema de feedback para incluir realidade virtual, permitindo aos clientes visualizarem e interagirem com as produções em um ambiente imersivo
    \item Implementação de um sistema automatizado de backup com redundância geográfica
\end{itemize}

Estas melhorias potenciais demonstram que, embora o sistema atual atenda às necessidades básicas do estúdio, existe ainda um grande potencial para evolução e aprimoramento das funcionalidades oferecidas.
